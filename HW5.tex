\documentclass[20pt]{article} % USenglish for autoref
\usepackage[paper=screen, centering, papersize=20cm]{geometry}
\usepackage{cmap}		% to search and copy ligatures
\usepackage[utf8]{inputenc}	% for Linux computer and Mac
%\usepackage[latin1]{inputenc}	% für Windows computer
\usepackage[T1]{fontenc}	% to search for ligatures in the pdf
\usepackage[USenglish]{babel}
\usepackage{amsmath}
\usepackage{amssymb}
\usepackage{enumerate}
\usepackage{amsthm}
\usepackage{graphicx}
%\usepackage{stmaryrd}		% for \mapsfrom
\usepackage{aliascnt}		% for Aliascounter so that autoref gives the thms the right names
\usepackage[pdfborder={0 0 0}]{hyperref}   % if you want to have math environment in captions you have to use \texorpdfstring{$x^2$}{x2}; without frame around the links
\usepackage[figure]{hypcap}		% to make autoref link to figures and not the captions of figures
%\usepackage{paralist}		% for compactitem
\usepackage{mathtools}		% for \coloneqq
\usepackage[]{todonotes}	% use option disable to disable all the todos
\usepackage{float}
\usepackage{lipsum}
\theoremstyle{break}
\newtheorem{definition}{Definition}[section]  
\newtheorem{exa}[definition]{Example}
\newtheorem{cor}[definition]{Corollary}
\newtheorem{lem}[definition]{Lemma}
\newtheorem{conj}[definition]{Conjecture}
\newtheorem{quest}[definition]{Research question}
\newtheorem{thm}[definition]{Theorem}  
\newtheorem{prop}[definition]{Proposition}
\newtheorem{rem}[definition]{Remark}


\renewcommand{\labelenumi}{(\roman{enumi})} % roman numbers in enumerations

\usepackage{graphicx,tikz} % Allows including images

\usetikzlibrary{calc,decorations.markings}
\usetikzlibrary{shapes,snakes}

\DeclareMathOperator{\Aff}{Aff}
\DeclareMathOperator{\der}{der}
\DeclareMathOperator{\Trans}{Trans}
\DeclareMathOperator{\dir}{dir}
\DeclareMathOperator{\Sing}{Sing}
\DeclareMathOperator{\Stab}{Stab}
\DeclareMathOperator{\id}{id}
\DeclareMathOperator{\im}{im}
\DeclareMathOperator{\NN}{\mathbb{N}}


%For L Cal and L Twidle:
\newcommand{\LC}{\mathcal{L}} 
\newcommand{\LT}{\widetilde{\mathcal{L}}}

%For X bar
\newcommand{\XB}{\overline{X}}

%For the real and complex numbers
\newcommand{\RR}{\mathbb{R}} 
\newcommand{\CC}{\mathbb{C}} 


\newcommand{\remark}[2][]{\todo[color=green!50, #1]{#2}}

\title{HW 5}


\begin{document}
\maketitle

1. (Problem 7.6 in textbook) Consider the trapezium and Simpson's rule applied to $\int_0^1(x^5-Cx^4)dx$.
\begin{itemize}
\item Write down the error for trapezium and Simpson's rule, as functions of $C$.
\item Find $C$ that makes the error under trapezium rule is $0$.
\item Find the range of $C$ where the trapezium rule is more accurate than Simpson's rule.
\end{itemize}

2. (Problem 7.11 in textbook) Suppose $f\in C^4([-1, 1])$. Let $p$ be the Hermite interpolation polynomial of $f$ at $x_0=-1$, $x_1=1$.
\begin{itemize}
\item Calculate $\int_{-1}^1pdx$ and write it as a linear combination of $f(\pm 1)$, $f'(\pm 1)$.
\item Prove that
  \[|\int_{-1}^1fdx-\int_{-1}^1pdx|\leq {2\over 45}\max_{c\in [-1, 1]}|f^{(4)}(c)|\]
\end{itemize}


3. (Problem 10.3 in textbook) Show that if $f\in C^2([0, 1])$, then there is some point $c\in (0, 1)$ such that
\[\int_0^1xfdx={1\over 2}f(2/3)+{1\over 72}f''(c)\]
Hint: use Gauss quadrature with weight $x$.\\

4. \begin{itemize}
\item Suppose $f$ is continuous on $[0, 1]$. Let $I_n$ be the estimate of $\int_0^1fdx$ using composite trapezium rule with $n$ subintervals. Show that
  \[\lim_{n\rightarrow\infty}|\int_0^1fdx-I_n|=0\]
 Hint: There are many possible approaches. You can use the fact that any continuous function on a closed interval is absolutely continuous, or use the Weierstrass approximation theorem.
\item (Optional) Find a continuous function $f$, such that there is $C>0$ such that 
  \[|\int_0^1fdx-I_n|\geq {C\over n}\]
  Hint: if $f$ has bounded second derivative then the error decays like $O(1/n^2)$, so you need to find some $f$ that doesn't have second order derivative or has unbounded second order derivative.
\end{itemize}
\end{document}