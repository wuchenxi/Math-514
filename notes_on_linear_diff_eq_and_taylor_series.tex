\documentclass[20pt]{article} % USenglish for autoref
\usepackage[paper=screen, centering, papersize=20cm]{geometry}
\usepackage{cmap}		% to search and copy ligatures
\usepackage[utf8]{inputenc}	% for Linux computer and Mac
%\usepackage[latin1]{inputenc}	% für Windows computer
\usepackage[T1]{fontenc}	% to search for ligatures in the pdf
\usepackage[USenglish]{babel}
\usepackage{amsmath}
\usepackage{amssymb}
\usepackage{enumerate}
\usepackage{amsthm}
\usepackage{listings}
\usepackage{graphicx}
\usepackage[]{algorithm2e}
%\usepackage{stmaryrd}		% for \mapsfrom
\usepackage{aliascnt}		% for Aliascounter so that autoref gives the thms the right names
\usepackage[pdfborder={0 0 0}]{hyperref}   % if you want to have math environment in captions you have to use \texorpdfstring{$x^2$}{x2}; without frame around the links
\usepackage[figure]{hypcap}		% to make autoref link to figures and not the captions of figures
%\usepackage{paralist}		% for compactitem
\usepackage{mathtools}		% for \coloneqq
\usepackage[]{todonotes}	% use option disable to disable all the todos
\usepackage{float}
\usepackage{lipsum}
\theoremstyle{break}
\newtheorem{definition}{Definition}[section]  
\newtheorem{exa}[definition]{Example}
\newtheorem{cor}[definition]{Corollary}
\newtheorem{lem}[definition]{Lemma}
\newtheorem{conj}[definition]{Conjecture}
\newtheorem{quest}[definition]{Research question}
\newtheorem{thm}[definition]{Theorem}  
\newtheorem{prop}[definition]{Proposition}
\newtheorem{rem}[definition]{Remark}


\renewcommand{\labelenumi}{(\roman{enumi})} % roman numbers in enumerations

\usepackage{graphicx,tikz,pgfplots} % Allows including images

\usetikzlibrary{calc,decorations.markings}
\usetikzlibrary{shapes,snakes}

\DeclareMathOperator{\Aff}{Aff}
\DeclareMathOperator{\der}{der}
\DeclareMathOperator{\Trans}{Trans}
\DeclareMathOperator{\dir}{dir}
\DeclareMathOperator{\Sing}{Sing}
\DeclareMathOperator{\Stab}{Stab}
\DeclareMathOperator{\id}{id}
\DeclareMathOperator{\im}{im}
\DeclareMathOperator{\NN}{\mathbb{N}}


%For L Cal and L Twidle:
\newcommand{\LC}{\mathcal{L}} 
\newcommand{\LT}{\widetilde{\mathcal{L}}}

%For X bar
\newcommand{\XB}{\overline{X}}

%For the real and complex numbers
\newcommand{\RR}{\mathbb{R}} 
\newcommand{\CC}{\mathbb{C}} 


\newcommand{\remark}[2][]{\todo[color=green!50, #1]{#2}}

\title{Notes on Linear Difference and Differential Equations and Taylor Series}


\begin{document}
\maketitle

\section{Linear Difference Equations}

A homogeneous linear difference equation is an iterative relationship:
\[z_{n+k}+a_{k-1}z_{n+k-1}+\dots+a_0z_n=0\]

The general solution of a homogeneous linear difference equation can be obtained as follows:
\begin{itemize}
\item Firstly, define the characteristic polynomial $\chi(z)=z^k+a_{k-1}z^{k-1}+\dots+a_0$.
\item Let $\lambda_1, \dots, \lambda_l$ be its distinct roots, $m_1, \dots, m_l$ their multiplicities (hence $\sum_im_i=k$).
\item Then, the general solution can be written as

\[z_n=\sum_ip_i(n)\lambda_i^n\]

Where $p_i(n)$ is any polynomial of degree no more than $m_i-1$.
\end{itemize}

\section{Linear Differential Equations}

Similarly, a homogeneous linear differential equation is
\[y^{(k)}+a_{k-1}y^{(k-1)}+\dots+a_1y'+a_0y=0\]

The general solution of a homogeneous linear differential equation is as follows:
\begin{itemize}
\item Firstly, define the characteristic polynomial $\chi(z)=z^k+a_{k-1}z^{k-1}+\dots+a_0$.
\item Let $\lambda_1, \dots, \lambda_l$ be its distinct roots, $m_1, \dots, m_l$ their multiplicities (hence $\sum_im_i=k$).
\item Then, the general solution can be written as

\[y(t)=\sum_ip_i(t)e^{\lambda_i t}\]

Where $p_i(n)$ is any polynomial of degree no more than $m_i-1$.
\end{itemize}


\section{Taylor Series}

If $f\in C^{k+1}$, then the Taylor series of $f$ at $a$, with Lagrange remainder, is

\[f(x)=f(a)+\sum_{j=1}^{k}{f^{(j)}(a)(x-a)^j\over j!}+{f^{(j+1)}(c)(x-a)^{j+1}\over(j+1)!}\] 

Where $c$ is in the closed interval between $x$ and $a$.

\end{document}