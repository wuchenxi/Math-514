\documentclass[20pt]{article} % USenglish for autoref
\usepackage[paper=screen, centering, papersize=15cm]{geometry}
\usepackage{cmap}		% to search and copy ligatures
\usepackage[utf8]{inputenc}	% for Linux computer and Mac
%\usepackage[latin1]{inputenc}	% für Windows computer
\usepackage[T1]{fontenc}	% to search for ligatures in the pdf
\usepackage[USenglish]{babel}
\usepackage{amsmath}
\usepackage{amssymb}
\usepackage{enumerate}
\usepackage{amsthm}
\usepackage{graphicx}
%\usepackage{stmaryrd}		% for \mapsfrom
\usepackage{aliascnt}		% for Aliascounter so that autoref gives the thms the right names
\usepackage[pdfborder={0 0 0}]{hyperref}   % if you want to have math environment in captions you have to use \texorpdfstring{$x^2$}{x2}; without frame around the links
\usepackage[figure]{hypcap}		% to make autoref link to figures and not the captions of figures
%\usepackage{paralist}		% for compactitem
\usepackage{mathtools}		% for \coloneqq
\usepackage[]{todonotes}	% use option disable to disable all the todos
\usepackage{float}
\usepackage{lipsum}
\theoremstyle{break}
\newtheorem{definition}{Definition}[section]  
\newtheorem{exa}[definition]{Example}
\newtheorem{cor}[definition]{Corollary}
\newtheorem{lem}[definition]{Lemma}
\newtheorem{conj}[definition]{Conjecture}
\newtheorem{quest}[definition]{Research question}
\newtheorem{thm}[definition]{Theorem}  
\newtheorem{prop}[definition]{Proposition}
\newtheorem{rem}[definition]{Remark}


\renewcommand{\labelenumi}{(\roman{enumi})} % roman numbers in enumerations

\usepackage{graphicx,tikz} % Allows including images

\usetikzlibrary{calc,decorations.markings}
\usetikzlibrary{shapes,snakes}

\DeclareMathOperator{\Aff}{Aff}
\DeclareMathOperator{\der}{der}
\DeclareMathOperator{\Trans}{Trans}
\DeclareMathOperator{\dir}{dir}
\DeclareMathOperator{\Sing}{Sing}
\DeclareMathOperator{\Stab}{Stab}
\DeclareMathOperator{\id}{id}
\DeclareMathOperator{\im}{im}
\DeclareMathOperator{\NN}{\mathbb{N}}


%For L Cal and L Twidle:
\newcommand{\LC}{\mathcal{L}} 
\newcommand{\LT}{\widetilde{\mathcal{L}}}

%For X bar
\newcommand{\XB}{\overline{X}}

%For the real and complex numbers
\newcommand{\RR}{\mathbb{R}} 
\newcommand{\CC}{\mathbb{C}} 


\newcommand{\remark}[2][]{\todo[color=green!50, #1]{#2}}


\begin{document}

5.11 Let the unknowns be $\lambda, x_1,\dots, x_n$, and the system of equations be $F(\lambda, x_1,\dots, x_n)=((A-\lambda I)x, x^Tx-1)=0$. Then by calculation the Jacobian matrix is
\[J=\left[\begin{array}{cc} A-\lambda I & -x \\ 2x^T & 0\end{array}\right]\]
Hence Newton's method gives us
\[\left[\begin{array}{c} x^{(1)} \\ \lambda^{(1)}\end{array}\right]=\left[\begin{array}{c} x^{(0)} \\ \lambda^{(0)}\end{array}\right]+\left[\begin{array}{c} \delta x \\ \delta\lambda\end{array}\right]\]
\[J\left[\begin{array}{c} \delta x \\ \delta\lambda\end{array}\right]=-\left[\begin{array}{c} (A-\lambda^{(0)}I)x^{(0)} \\ (x^{(0)})^Tx^{(0)}-1\end{array}\right]\]
Hence
\[(A-\lambda^{(0)}I)\delta x-\delta\lambda x^{(0)}=-(A-\lambda^{(0)}I)x^{(0)}\]
\[-{x^{(0)}}^T\delta x={1\over 2}({x^{(0)}}^Tx^{(0)}-1)\]
It's easy to see that the direction of eigenvectors changes in the same way as inverse iteration, the difference being that at each step their norms are not strictly normalized into $1$.

\end{document}