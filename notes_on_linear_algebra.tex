\documentclass[20pt]{article} % USenglish for autoref
\usepackage[paper=screen, centering, papersize=15cm]{geometry}
\usepackage{cmap}		% to search and copy ligatures
\usepackage[utf8]{inputenc}	% for Linux computer and Mac
%\usepackage[latin1]{inputenc}	% für Windows computer
\usepackage[T1]{fontenc}	% to search for ligatures in the pdf
\usepackage[USenglish]{babel}
\usepackage{amsmath}
\usepackage{amssymb}
\usepackage{enumerate}
\usepackage{amsthm}
\usepackage{graphicx}
%\usepackage{stmaryrd}		% for \mapsfrom
\usepackage{aliascnt}		% for Aliascounter so that autoref gives the thms the right names
\usepackage[pdfborder={0 0 0}]{hyperref}   % if you want to have math environment in captions you have to use \texorpdfstring{$x^2$}{x2}; without frame around the links
\usepackage[figure]{hypcap}		% to make autoref link to figures and not the captions of figures
%\usepackage{paralist}		% for compactitem
\usepackage{mathtools}		% for \coloneqq
\usepackage[]{todonotes}	% use option disable to disable all the todos
\usepackage{float}
\usepackage{lipsum}
\theoremstyle{break}
\newtheorem{definition}{Definition}[section]  
\newtheorem{exa}[definition]{Example}
\newtheorem{cor}[definition]{Corollary}
\newtheorem{lem}[definition]{Lemma}
\newtheorem{conj}[definition]{Conjecture}
\newtheorem{quest}[definition]{Research question}
\newtheorem{thm}[definition]{Theorem}  
\newtheorem{prop}[definition]{Proposition}
\newtheorem{rem}[definition]{Remark}


\renewcommand{\labelenumi}{(\roman{enumi})} % roman numbers in enumerations

\usepackage{graphicx,tikz} % Allows including images

\usetikzlibrary{calc,decorations.markings}
\usetikzlibrary{shapes,snakes}

\DeclareMathOperator{\Aff}{Aff}
\DeclareMathOperator{\der}{der}
\DeclareMathOperator{\Trans}{Trans}
\DeclareMathOperator{\dir}{dir}
\DeclareMathOperator{\Sing}{Sing}
\DeclareMathOperator{\Stab}{Stab}
\DeclareMathOperator{\id}{id}
\DeclareMathOperator{\im}{im}
\DeclareMathOperator{\NN}{\mathbb{N}}


%For L Cal and L Twidle:
\newcommand{\LC}{\mathcal{L}} 
\newcommand{\LT}{\widetilde{\mathcal{L}}}

%For X bar
\newcommand{\XB}{\overline{X}}

%For the real and complex numbers
\newcommand{\RR}{\mathbb{R}} 
\newcommand{\CC}{\mathbb{C}} 


\newcommand{\remark}[2][]{\todo[color=green!50, #1]{#2}}

\title{Notes on Linear Algebra}


\begin{document}
\maketitle

Recall that a (real) {\bf Vector space} $V$ is a set with an element $0$, a ``scalar multiplication'' map $\mathbb{R}\times V\rightarrow V$ and an ``addition'' map $V\times V\rightarrow V$, such that, for any $x, y, z\in V$, any $a, b\in \mathbb{R}$, the followings are true:
\begin{enumerate}
\item $x+y=y+x$
\item $(x+y)+z=x+(y+z)$
\item $0+x=0$
\item $1x=x$
\item $0x=0$
\item $(a+b)x=ax+bx$
\item $a(x+y)=ax+ay$
\item $(ab)x=a(bx)$
\end{enumerate}

If $V$ is a vector space, any non empty subset $V'\subset V$ which is closed under addition and scalar multiplication is called a {\bf subspace}. The maximum number of linearly independent vectors in a vector space is called its {\bf dimension}.\\

The set of $n$ dimensional column vectors $\mathbb{R}^n$, under the usual addition and scalar multiplication, is a vector space, and it has dimension $n$.\\

The {\bf inner product} on $\mathbb{R}^n$ is defined as $(x, y)=x^Ty=\sum_ix_iy_i$. It is easy to check that this inner product satisfies the following properties:
\begin{enumerate}
\item Symmetry: $(x, y)=(y, x)$
\item Bilinearity: $(ax+a'x', y)=a(x, y)+a'(x', y)$, $(x, by+b'y')=b(x, y)+b'(x, y')$.
\item Positive definiteness: $(x, x)\geq 0$ and $(x, x)=0$ iff $x=0$.
\end{enumerate}

Let $V$ be a subspace of $\mathbb{R}^n$. We call a basis of $V$ {\bf orthogonal} if the inner product of distinct basis vectors are all $0$, {\bf orthonomal} if in addition, the inner product of any basis vector with itself is $1$.\\

Given any basis $\{x_1, \dots, x_d\}$ of a subspace $V\subset \mathbb{R}^n$, we can make it into an orthogonal or orthonormal basis via the {\bf Gram-Schmidt process}:
\[y_1=x_1\]
\[y_i=x_i-\sum_{j<i}((y_j, x_i)/(y_j, y_j))y_j\]
Then $\{y_i\}$ is an orthogonal basis, and $\{(y_i, y_i)^{-1/2}y_i\}$ is an orthonormal basis.\\

If $V$ is a subspace of $\mathbb{R}^n$, $x\in \mathbb{R}^n$, we call the {\bf orthogonal projection} of $x$ on $V$, denoted as $P_V(x)$, the unique vector that satisfies $P_V(x)\in V$ and $(x-P_V(x), y)=0$ for all $y\in V$.\\

For any $x'\in V$, $(x-P_V(x), x-P_V(x))\leq (x-x', x-x')$ and equality happens iff $x'=P_V(x)$.\\

To calculate $P_V(x)$, we can use either of these formulas:

\begin{enumerate}
\item If $\{x_i\}$ is an orthonormal basis of $V$, then $P_V(x)=\sum_i(x, x_i)x_i$.
\item If $\{x_i\}$ is an orthogonal basis of $V$, then $P_V(x)=\sum_i((x, x_i)/(x_i, x_i))x_i$.
\item If $\{x_i\}$ is just a basis of $V$, let $X=[x_1, \dots x_d]$ be a $n\times d$ matrix, then
  \[P_V(x)=X(X^TX)^{-1}X^Tx=\sum_i(\sum_j a_{ij}(x_j, x)) x_i\]
  Where $A=[a_{ij}]=[(x_i, x_j)]^{-1}$ is a $d\times d$ matrix. 
\end{enumerate}

If one replace $(x, y)$ with $(x, y)_A$ defined as $x^TAy$, where $A$ is a symmetric matrix with all eigenvalues positive, then $(\cdot, \cdot)_A$ still satisfies symmetry, bilinearity and positive definiteness, and all the conclusions about $(\cdot, \cdot)$ above are still valid.

\end{document}